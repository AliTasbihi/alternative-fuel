
%%%  تالار گفتگوی پارسی‌لاتک،       http://forum.parsilatex.com


%        روش اجرا.: 2 بار F1 ، 2 بار  F11(به منظور تولید مراجع) ، دوبار Ctrl+Alt+I (به منظور تولید نمایه) و دو بار F1 -------> مشاهده Pdf
%%%%%%%%%%%%%%%%%%%%%%%%%%%%%%%%%%%%%%%%%%%%%%%%%%%%%%
%   TeXstudio as your IDE
%%  برای compile در TeXstudio تنها کافی است منوی Options->Configure TeXstudio را زده و در پنجره Configure TeXstudio در بخش Build گزینه Default Compiler را به XeLaTeX تغییر دهید. سند شما به راحتی compile خواهد شد.
%   F1 & F5 : Build & view
%   F6      : Compile
%   F7      : View
%   --------------
%%%%%%%%%%%%%%%%%%%%%%%%%%%%%%%%%%%%%%%%%%%%%%%%%%%%%%
%%% !TEX TS-program = XeLaTeX
\documentclass[oneside,msc,12pt]{main}
\usepackage{ragged2e}
\newcommand\pd[2]{\displaystyle\frac{\partial #1}{\partial #2}}

\usepackage[acronym]{glossaries}
\usepackage{calc}
%\usepackage{caption_2010-01-09}
\usepackage{array}
\usepackage{makecell}
\usepackage{multirow}
\usepackage{longtable}
%\usepackage{graphicx}
%\usepackage{array}
\usepackage{rotating}

\usepackage{lscape}
\usepackage{pdflscape}
\usepackage{wasysym}
%\usepackage{arabicore}
%\usepackage{caption_2019-09-01}
\makeglossary
\newacronym{co}{CO}{heavy-industrit-sd}
%\usepackage{bit}
%\usepackage{natbib}
%\addbibresource{references.bib}
\usepackage{forest}
\usepackage{fontspec}


\usepackage{tikz}
\usetikzlibrary{arrows,automata}
\usetikzlibrary{shapes,arrows,trees,calc}
\input{commands}
% for numbering subsubsections
\setcounter{secnumdepth}{3}
%to include subsubsections in the table of contents
\setcounter{tocdepth}{3}

\begin{document}
\baselineskip=.75cm
\linespread{1.75}
\input{fa_title}
% دستور زیر برای شماره گذاری صفحات قبل از فصل اول با حروف ابجد است.
\pagenumbering{alph}
%-----------------------------------------------------------------------------
% فایل زیر دستورات مربوط به نمایش صفحات فهرست مطالب- فهرست اشکال و جداول است.
\input{TOC-TOF-LOT}
%-------------------------------------------------------------------------symbols(فهرست نمادها)
\input{list-of-symbols}

\pagenumbering{arabic}
\pagestyle{style1}
%--------------------------------------------------------------------------chapters(فصل ها)
%\begin{center}
	
\end{center}\chapter{سوخت \lr{LNG}}
\section{کلیات}
گاز طبیعی مایع یا ال‌ان‌جی 
\LTRfootnote{Liquefied Natural Gas}
 به گاز طبیعی‌ای گفته می‌شود که موقتاً برای ذخیره‌سازی یا ترابری در حجم بالا، به حالت مایع تبدیل شده‌است.
 \\
 ال‌ان‌جی، بیشترشامل متان 
 \LTRfootnote{(CH4)} 
 و مقادیر اندکی اتان، پروپان، بوتان و برخی از آلکان‌های سنگین دیگر می‌باشد. البته در مواردی خاص، فرایند پالایش ال‌ان‌جی، می‌تواند به نحوی طراحی شود، که محصول پایانی تولیدشده شامل ۱۰۰ درصد متان باشد. 
 \lr{LNG}
  به عنوان پاک ترین سوخت فسیلی شناخته شده و اقبال به استفاده از این سوخت رو به افزایش است طوری که طبق آمار شناورهای سفارش داده شده برای ساخت در سال 2019، حدود
   17\%  از این شناورها (بر حسب تناژ) از سوخت 
 \lr{LNG} 
   بهره‌مند خواهند شد.
ال‌ان‌جی حجمی معادل یک ششصدم حجم گاز طبیعی در حالت گازی را دارا بوده و محلولی بی‌بو، بی‌رنگ، غیرسمی و غیرخورنده است. این سوخت دارای چگالی کمتری در مقایسه با سوخت‌های رایج بوده و به همین دلیل برای ذخیره سازی نیاز به مخازن بزرگتری دارد.
\begin{figure}[!h]
	\centering
	\includegraphics[width=15cm]{Figures/LNG/Fuel Type.png}
	\caption{انواع سوخت}\label{fuel type}
\end{figure}
\newpage
\section{اثرات زیست محیطی و ایمنی}
\subsection{تأثیرات مثبت}
\begin{enumerate}
	\item  کاهش15 تا 20 درصدی انتشار گازهای گلخانه‌ای \lr{(CO2)} به دلیل نسبت بالای هیدروژن به کربن در مقایسه با سایر سوخت‌ها
	\item کاهش 100 درصدی انتشار اکسیدهای گوگرد \lr{(SOx)} در مقایسه با سوخت‌های فعلی
	\item کاهش بالای 90 درصدی انتشار اکسیدهای نیتروژن \lr{(NOx)} در مقایسه با سوخت‌های فعلی
	\item کاهش 100 درصدی انتشار ذرات معلق \lr{(PM)} در مقایسه با سوخت‌های فعلی
\end{enumerate}
\subsection{تأثیرات منفی}
\begin{enumerate}
	\item 	تأثیرات زیست‌محیطی هنگام استخراج، فرآوری، حمل، ذخیره سازی و مصرف نهایی \lr{LNG}
	\item احتمال نشت متان \LTRfootnote{Methane Slip} در موتور شناور به ویژه در فرایندهای احتراق ناقص که بعضا اثرات گلخانه‌ای شدیدتری نسبت به دی اکسید کربن خواهد داشت.
	\item احتمال نشت سوخت مایع در آب دریا و تبخیر منجر به آلودگی هوا

\end{enumerate}

\section{تکنولوژی تولید}
\begin{figure}[!h]
	\centering
	\includegraphics[width=15cm]{Figures/LNG/technology.png}
	\caption{تکنولوژی تولید}\label{technology}
\end{figure}
\subsection{مرحله اول، پیش تصفیه سازی}
ابتدا خوراک گاز طبیعی در واحد زدایش گازهای اسیدی از میان فیلترها عبور نموده و گازهای اسیدی مانند
\lr{CO2} و
\lr{H2S}   
با استفاده از یک محلول آمین اختصاصی در ستون جذب جدا می‌شوند.
سپس واحد آب زدایی گاز طبیعی را تا مقدار مجاز کمتر از 0.1
 \lr{ppm}
  از آب خشک نموده و جریان در ادامه مسیر خود وارد واحد جیوه زدایی می شود. زیرا برای ورود خوراک به واحد مایع سازی باید کمترین میزان جیوه نیز از این ترکیب حذف گردد.
\subsection{مرحله حذف هیدروکربن های سنگین}
پس از یک مرحله خنک سازی ابتدایی با آمونیاک به عنوان مبرد (تا دمای 
8-
 درجه سانتیگراد)، حذف هیدروکربن های سنگین مانند پنتان و بنزن از خوراک گازی صورت می‌گیرد. مقدار مجاز این مواد برای جلوگیری از یخ زدگی خوراک در واحد پایین دستی کمتر از 1 
\lr{ppm}
  است.
\subsection{مرحله مایع سازی}
در این مرحله گاز ورودی توسط سیتم‌های تبرید تحت فشار از دمای 8- درجه سانتی به 162- سانتی گراد خنک گردیده و تبدیل به مایع می شود. در انتها گاز طبیعی مایع شده به مخازن نگهداری ارسال می شود. در بازارهای مصرف، 
\lr{LNG}
 دوباره به حالت گاز در می‌آید. 
\section{تکنولوژی استفاده در کشتی}
\begin{figure}[!h]
	\centering
	\includegraphics[width=15cm]{Figures/LNG/using in Ship.png}
	\caption{تکنولوژی استفاده در کشتی}\label{using in ship}
\end{figure}
به طور کلی دو گزینه برای استفاده از 
\lr{LNG}
 به عنوان سوخت در کشتی وجود دارد:
 \begin{enumerate}
 	\item استفاده در موتورهای احتراق داخلی \lr{(ICE)}
 	\item استفاده در پیل‌های سوختی  \lr{(Fuel Cell)}
 \end{enumerate}
 به طور کلی سه نوع موتور برای استفاده از 
 \lr{LNG}
  به عنوان سوخت مورد استفاده هستندکه هر سه دوگانه سوز بوده و علاوه بر 
  \lr{LNG}
   از سوخت‌های رایج مانند مازوت هم بهره می‌برند:
    \begin{enumerate}
   	\item موتور دو زمانه فشار پایین
   	\item موتور دو زمانه فشار بالا (با احتمال انتشار \lr{NOx})
   	\item موتور چهارزمانه فشار پایین
   \end{enumerate}
   در ادامه به پیل‌های سوختی پرداخته خواهد شد.
  \subsection{الزامات قانونی}
  علاوه بر قوانین و الزامات محیط زیستی سازمان \lr{IMO} برای کاهش انتشار گازهای گلخانه‌ای \lr{(CO2)}، اکسیدهای گوگرد \lr{(SOx)} واکسید نیتروژن \lr{(NOx)} که می‌تواند میزان گرایش به استفاده از سوخت‌های جایگزین مانند \lr{LNG} را افزایش دهد، الزامات ایمنی کشتی‌های استفاده کننده از سوخت‌های گازی یا سوخت‌های دارای نقطه اشتعال پایین \LTRfootnote{(IGF Code)} به طور مشخص به \lr{LNG} پرداخته و نکات ایمنی برای طراحی و ساخت شناورهای مصرف کننده \lr{LNG} مطرح نموده است. این الزامات شامل سه موضوع اصلی است که عبارتند از:
  \begin{enumerate}
  	\item پیشگیری از نشت سوخت
  	\item پیشگیری از انفجار یا فضای سمی
  	\item مقابله و کاهش اثرات انفجار 
  \end{enumerate}
  الزامات جنبه‌های دیگر مصرف \lr{LNG} به عنوان سوخت دریایی مانند فرایند سوخت رسانی \LTRfootnote{(Bunkering)} هم تا این لحظه به قوانین و مقررات داخلی کشورها واگذار گردیده است.  علاوه بر \lr{IMO} ، اتحادیه اروپا هم الزامات محدودکننده‌ای در راستای کاهش انتشار گازهای گلخانه‌ای و سایر آلاینده‌های هوا وضع نموده است 
\LTRfootnote{(FuelEU Maritime)}.
\section{بررسی اقتصادی \lr{LNG}}
\subsection{قیمت}
	در اکثر نقاط جهان قیمت گاز طبیعی پایین تر از قیمت نفت خام و سوخت‌های سنگین مانند مازوت است. قیمت \lr{LNG} هم ارتباط مستقیمی با قیمت گاز طبیعی دارد که البته هزینه‌های مایع سازی، حمل و نقل و توزیع، ذخیره سازی و سود ذینفعان هم به آن اضافه ‌می‌شود.
در اواسط دهه گذشته میلادی (حدود سال‌های 2015 و 2016) با اکتشاف و بهره‌برداری از منابع جدید گاز طبیعی در امریکا و کاهش قیمت جهانی نفت خام، شاهد کاهش چشمگیر قیمت گاز طبیعی در اکثر نقاط جهان بودیم که طبیعتا کاهش قابل توجه قیمت \lr{LNG} را هم به همراه داشت. از سال 2016 به بعد قیمت \lr{LNG} با شیب ملایمی در حال رشد بوده و پیش بینی می‌شود این روند افزایشی آرام همچنان ادامه دار باشد، هرچند درگیری نظامی روسیه و اوکراین در سال 2022، با ایجاد شوک اقتصادی، باعث افزایش ناگهانی قیمت گاز شد اما با گذشت چند ماه شرایط به یک ثبات نسبی رسید.
در حال حاضر میزان رشد قیمت \lr{LNG} در مقایسه با گازوئیل \lr{(MGO) }و سوخت‌های سنگین کم سولفور کمتر بوده این روند \lr{LNG} را به عنوان یک سوخت با قیمت رقابتی در بازار و یک گزینه مقرون به صرفه در آینده نزدیک مطرح می‌نماید. به ویژه که قوانین زیست محیطی سختگیرانه بعدی هم در راه خواهند بود.
هم اکنون (تابستان 2024) قیمت \lr{LNG} به ثبات نسبی رسیده و اغلب از گازوئیل پایین تر ولی همچنان از سوخت‌هایی مثل مازوت گران تر است اما با توجه به محدودیت‌های زیست محیطی مثل محدودیت سولفور نیم درصد، احتمالا تولید و عرضه مازوت در آینده به صرفه نخواهد بود. 
\subsection{هزینه‌های اولیه}
هزینه‌های اولیه راه‌اندازی سیستم‌های \lr{LNG} بر روی کشتی شامل هزینه موتور، هزینه مخازن و ذخیره سازی، هزینه سیستم‌های آماده سازی و انتقال سوخت و هزینه ارتقا سیستم‌های قبلی می‌شود. در حال حاضر هزینه‌های اولیه سیستم‌های \lr{LNG} (به خصوص به دلیل حجم مخازن بزرگتر و نیاز به عایق بندی به سبب نقطه جوش پایین تر) در مقایسه با سوخت‌های رایج مانند مازوت و گازوئیل بالاتر است اما با توجه افزایش توجهات به سمت \lr{LNG} و همچنین افزایش سرمایه‌گذاری و توسعه فناوری‌های مرتبط، انتظار می‌رود در سال‌های آینده شاهد کاهش نسبی این هزینه‌ها باشیم. در حال حاضر تخمین زده می‌شودکه بااستفاده از سوخت \lr{LNG} در کشتی‌ها، متوسط هزینه‌های اولیه در حدود 20 درصد افزایش پیدا کند. 
\subsection{هزینه‌های ثانویه (عملیاتی)}
در صورتی که قیمت \lr{LNG} در سطح جهانی جهش ناگهانی نداشته باشد، می‌توان هزینه‌های عملیاتی سیستم‌های \lr{LNG} را در مقایسه با سیستم‌های سوخت‌های فعلی، برابر و یا حتی کمتر دانست. چرا که سیستم‌های \lr{LNG} اغلب نیاز به سیستم‌های کاهش انتشار و اسکرابر ندارند و بازدهی و مصرف سوخت موتورهای مصرف کننده \lr{LNG} (دوگانه سوز) تفاوتی با موتورهای قبلی ندارد. همچنین به دلیل پاک تر بودن سوخت \lr{LNG} می‌توان انتظار داشت، هزینه‌های تعمیر و نگهداری موتورها هم در مقایسه با سوخت‌های سنگین کمتر باشد. در ضمن در برخی بنادر به عنوان مشوق تخفیف‌هایی برای شناورهای دارای سوخت \lr{LNG} در نظر می‌گیرند. 
\subsection{میزان تولید، زیرساخت‌ها و میزان دسترسی}
در ابتدای سال 2024، ظرفیت سالانه تولید \lr{LNG}، در حدود 483.1 میلیون تن، تخمین زده می‌شود و با توجه به سرمایه‌گذاری‌های صورت گرفته در نقاط مختلف جهان، انتظار می‌رود این عدد تا سال 2030 به حدود 700 میلیون تن در سال برسد.
همچنین تا پایان سال 2023، 49 ترمینال فعال عرضه کننده سوخت \lr{LNG} دریایی (ساحلی یا فراساحلی) در سراسر جهان در حال خدمت رسانی هستند که مجموع ظرفیت سالانه عرضه سوخت آنها حدود 200 میلیون تن می‌باشد و حداقل 17 ترمینال جدید هم در حال ساخت بوده و به زودی به این ظرفیت اضافه خواهند شد.
اولین بنادری که شروع به عرضه سوخت \lr{LNG} به شناورها نمودند آمستردام و روتردام هلند، فلوریدا امریکا و بنادری در دریای شمال و دریای بالتیک بودند. اما کم کم سایر نقاط جهان مانند بنادر مدیترانه غربی، خلیج مکزیک، سنگاپور، استرالیا، چین، ژاپن، کره جنوبی و حتی بنادر کشورهای خاورمیانه مثل امارات هم سرمایه‌گذاری و بهره برداری از مراکز تامین \lr{LNG} را آغاز نمودند. 
در حال حاضر سوخت \lr{LNG} سهم کمتر از 5 درصدی از بازار سوخت‌های دریایی دارد که احتمالا این عدد در سالهای آینده افزایش قابل توجهی خواهد داشت. با توجه به افزایش روز افزون تولید جهانی \lr{LNG} و روش‌های مختلف توزیع  سوخت در سراسر جهان (حمل دریایی، زمینی و حتی ریلی) می‌توان این گونه فرض کرد که در آینده نزدیک از بابت میزان دسترسی به این سوخت چالش جدی نخواهیم نداشت.
\section{وضعیت ایران}
\subsection{تاریخچه تولید \lr{LNG} در ایران}
ایران با دارا بودن دومین ذخایر بزرگ گاز طبیعی جهان، پتانسیل قابل توجهی برای تبدیل شدن به یک بازیگر کلیدی در بازار \lr{LNG} جهانی را دارد. با این حال، ایران تاکنون در تولید \lr{LNG} موفقیت چندانی کسب نکرده است.
\begin{enumerate}
	\item دهه 1970: اولین مطالعات برای احداث واحدهای \lr{LNG} در ایران انجام شد.
	\item دهه 1990: به دلیل تحریم‌ها و عدم ثبات سیاسی، پیشرفت‌ها متوقف شد.
	\item دهه 2000: چندین پروژه \lr{LNG} با مشارکت شرکت‌های خارجی آغاز شد، اما هیچکدام به بهره‌برداری نرسیدند.
	\item دهه 2010: ایران به دنبال توسعه مجدد پروژه‌های \lr{LNG} با تمرکز بر فناوری‌های کوچک مقیاس است.
\end{enumerate}
ایران در حال ساخت یک کارخانه تولید \lr{LNG} با ظرفیت 1.5 میلیون تن در سال در عسلویه است. این کارخانه توسط یک شرکت ایرانی با هدف افزایش مصارف داخلی و همچنین صادرات احداث گردیده است. فاز اول این کارخانه شامل نیروگاه 1100 مگاواتی، مخازن ذخیره‌سازی \lr{LNG} و \lr{LPG}، و اسکله‌های دریایی است. 
ایران برای تولید \lr{LNG} به فناوری پیشرفته ای نیاز دارد که ممکن است به دلیل تحریم ها در دسترس نباشد. ساخت واحدهای \lr{LNG} نیاز به سرمایه گذاری عظیمی دارد. تحریم های بانکی و اقتصادی ایران تامین مالی پروژه‌های \lr{LNG} را با مشکل مواجه می کند. ایران با رقابت شدیدی در بازار \lr{LNG} از دیگر کشورها مانند قطر، استرالیا و ایالات متحده مواجه است.
با افزایش تولید \lr{LNG} در کشور، امکان عرضه آن به عنوان سوخت دریایی در بنادر ایران فراهم خواهد شد.

$\pd{s}{d}$
\subsection{پیش بینی تقریبی از آینده}
پیش بینی ها نشان می‌دهند که گاز طبیعی مایع \lr{(LNG)} تا سال ۲۰۳۰ به عنوان یک سوخت به صرفه در صنعت حمل و نقل دریایی باقی خواهد ماند. در حال حاضر، \lr{LNG}  حدود 6.4 درصد از سوخت های مورد نیاز صنعت کشتیرانی را تأمین میکند و انتظار میرود که این میزان تا سال ۲۰۳۰ به ۱۰.۷ درصد افزایش یابد.
همچنین، با توجه به قوانین سختگیرانه سازمان بین المللی دریانوردی \lr{(IMO)} برای کاهش انتشار گازهای     گلخانه ای، استفاده از \lr{LNG} به عنوان یک سوخت پاک و پایدار تشویق خواهد شد.
در آینده، با توسعه زیرساخت های ذخیره سازی و سوخت رسانی \lr{LNG}، تعداد کشتیهایی که از این سوخت استفاده میکنند به طور قابل توجهی افزایش خواهد یافت. پیش بینی میشود که بیش از ۱۰۰۰ فروند کشتی با سوخت \lr{LNG} تا سال ۲۰۲۷ در آبهای بین المللی فعالیت کنند.
\section{نتیجه‌گیری}
\begin{figure}[!h]
	\centering
	\includegraphics[width=20cm]{Figures/LNG/conv.png}
	\caption{جمع بندی}\label{conv}
\end{figure}




%\chapter{سوخت \lr{LPG}}
\section{ مقدمه و اهمیت \lr{LPG} در حمل و نقل دریایی}
تا دسامبر ۲۰۲۳، تعداد ۱۶۴۴ کشتی حامل
 \lr{LPG (LPGC)}
 وجود داشت که از میان آن‌ها، ۲۰۴ کشتی، بر اساس مدل موتور اصلی نصب‌شده، قابلیت تبدیل به سوخت 
 \lr{LPG }
 را داشتند. این تغییر امکان بهره‌برداری از محموله کشتی‌ها و زیرساخت‌های موجود را فراهم کرده و هزینه‌های عملیاتی آن‌ها را به حداقل می‌رساند. 
به عنوان یک سوخت 
\lr{LPG} 
منحصر به فرد هست،‌ زمانی که از کربن‌‌زدایی در کشتی صحبت می‌کنیم نقش مهم و در حال رشدی 
\lr{LPG}
بازی می‌کند.
\lr{LPG} 
عمدتاً از پروپان و بوتان تشکیل شده که تفاوت‌های جزئی شیمیایی آن‌ها باعث کاربردهای خاص می‌شود.
\lr{LPG}
می‌تواند تحت فشار متوسط در دمای معمول مایع شود، که حمل‌ونقل و ذخیره‌سازی آن را آسان‌تر از سایر سوخت‌های گازی می‌کند.
سوخت 
\lr{LPG} 
به صورت مایع حمل و نگهداری می‌شود، اما به صورت گاز مصرف می‌شود، در حالیکه می‌تواند حمل و نقل دریایی تمیزتر به نسبت  بسیاری از جایگزین‌های موجود در حال حاضر ارائه دهد.

سفارشات کشتی‌های با سوخت 
\lr{LPG}
 رکورد شکن شده است، به طور مثال کشتی‌های
\lr{117 VLGC}
  سفارش داده شده یا در حال ساخت با
\lr{LPG}
   حرکت می‌کنند. و پیش‌بینی می‌شود که 86 درصد از این نوع کشتی‌های  
     جدیدی که در سال‌های آینده وارد بازار می‌شوند، قابلیت کار با 
\lr{LPG}
    را داشته باشند. اگرچه 
 \lr{LPG} 
    در حال حاضر یک سوخت محبوب برای حامل های گاز بزرگ است، این بخش تنها 8 درصد از آلاینده های حمل و نقل را به خود اختصاص می‌دهد و 92 درصد آلاینده‌ها را بقیه کشتی‌ها به جا می‌گذارند. تبدیل این کشتی ها یک فرصت عالی برای کاهش انتشار گازهای گلخانه‌ای جهانی است.
جهان ما به طور فزاینده‌ای به سمت کربن کم‌تر حرکت می‌کند و همه بخش‌های اقتصاد باید به مسئله انتشار گازهای گلخانه‌ای بپردازند.
\newpage

در بخش کشتیرانی،
دلایل استفاده از این سوخت به شرح زیر هست:
\begin{enumerate}
	\item 
	توسعه 
	\lr{LPG}
	 به عنوان سوخت دریایی به گسترش فناوری‌های کاهش کربن وابسته است، و پیش‌بینی می‌شود،
	  \lr{rLPG }
	  \LTRfootnote{Refrigerated Liquefied Petroleum Gas}
	  تا سال 2050 تا 50 درصد تقاضای جهانی را پوشش دهد.
	\item 
	\lr{LPG}
	 با استانداردهای زیست‌محیطی فعلی از جمله \textbf{محدودیت گوگرد} سازمان بین‌المللی دریانوردی 
	 \lr{IMO}
	  مطابقت دارد.
	\item 
	\lr{LPG} 
	دارای شبکه حمل‌ونقل گسترده‌ای است که شامل بیش از 1600 کشتی حامل 
	\lr{LPG}
	 و بیش از 1000 تأسیسات ذخیره‌سازی است.
	\item  
	می‌تواند عملکرد زیست‌محیطی بخش کشتیرانی را به سرعت بهبود بخشد.
	\item 
	انتشار گازهای مضر آن کم و هزینه آن، مقرون‌به‌صرفه است که به بهبود محیط‌زیست کمک می‌کند.
	\item 
	\lr{LPG}
	 انعطاف‌پذیر است و زنجیره‌های تأمین آن در سراسر جهان موجود است، که باعث می‌شود زیرساخت‌های سوخت‌رسانی راحت‌تر از بسیاری از سوخت‌های جایگزین دیگر پیاده‌سازی شود.
	 \item 
	 برای سیستم‌های پیشران مبتنی بر
	  \lr{LPG}
هیچ محدودیت یا مانع فناورانه‌ای وجود ندارد.
	
	  \lr{LPG}
	  به فناوری جدید یا پیشرفته نیاز ندارد و آماده بهره‌برداری است.
	 \item 
	 چه برای بزرگ‌ترین کشتی‌های جهان و چه برای کوچک‌ترین موتورهای قایق،
	  \lr{LPG} 
	  امروز یک سوخت کم‌کربن و کم‌انتشار ارائه می‌دهد، و با معرفی 
	  \lr{LPG} 
	  تجدیدپذیر، کربن‌زدایی کم‌هزینه در آینده امکان‌پذیر می‌شود.
\end{enumerate}
\cite{LR_LPG}
	 
\newpage

\begin{table}[h!]
	\centering
	\caption{مشخصات سوخت \lr{LPG}}
	\label{dsd}
	\begin{tabular}{|m{2cm}|m{12cm}|}
		\hline
		\textbf{مشخصه} & \textbf{توضیحات} \\
		\hline
		دمای جوش   & $-42$ درجه (پروپان خالص) و $-0.5$ درجه (بوتان خالص) \\
		فشار بخار & $1.8$ بار (بوتان خالص) تا $7.3$ بار (پروپان خالص) در دمای 15 درجه\\
		 چگالی & $1.89$ کیلوگرم بر متر مکعب (پروپان خالص) تا $2.54$ کیلوگرم بر متر مکعب (بوتان خالص) در دمای $15$ ‌درجه \\
		حداقل انرژی احتراق & $ 0.25 mj$\\
		چگالی انرژی حجمی &  $  26.5MJ/L $   \\
		نسبت اندازه مخزن & $1.5$  \\
		\hline
	\end{tabular}
\end{table}

\section{اثرات زیست محیطی}
\section{تکنولوژی‌های مرتبط با \lr{LPG}}
\subsection{تکنولوژی تولید}
\subsection{تکنولوژی استفاده در کشتی}
\subsection{ایمنی و الزامات فنی }
ذخیره‌سازی، استفاده و حمل‌ونقل
 \lr{LPG}
 خطرات بالقوه‌ای را به همراه دارد که باید در تمامی سناریوهای صنعتی کاهش یابد. در زمینه سوخت دریایی، برای مقابله با این خطرات، اجرای تدابیری در طراحی و ساخت کشتی، تنظیمات ماشین‌آلات و فناوری‌ها، فناوری‌های سوخت‌رسانی، رویه‌های داخلی کشتی و آموزش خدمه ضروری است.
\subsubsection{سنگینی نسبت به هوا}
\lr{LPG}
به صورت گازی تقریباً دو برابر سنگین‌تر از هوا است. این ویژگی باعث می‌شود که در سطوح پایین‌تر جمع شود و خطراتی را در مکان‌های بسته یا گودال‌ها ایجاد کند.تهویه مناسب در سطوح پایین و فضاهای بسته الزامی است.
همچنین استفاده از آشکارسازهای گاز در نزدیکی زمین برای شناسایی نشت گاز الزامی هست.
\subsubsection{مخلوط قابل اشتعال  با هوا}
\lr{LPG}
در غلظت 2 تا 10 درصد با هوا مخلوط قابل اشتعالی تشکیل می‌دهد. در صورت ذخیره یا استفاده نادرست، خطر آتش‌سوزی و انفجار وجود دارد.دوری از منابع گرما، جرقه و شعله باز.
استفاده از تجهیزات ضد انفجار در محل ذخیره یا استفاده.
\subsubsection{تأثیرات استنشاق در غلظت‌های بالا}
در غلظت‌های بسیار بالا، 
\lr{LPG}
 می‌تواند اثرات بی‌هوشی و خفگی داشته باشد زیرا اکسیژن موجود در هوا را رقیق می‌کند.اطمینان از وجود تهویه کافی در فضاهای بسته.
 استفاده از ماسک‌های تنفسی مناسب در شرایط اضطراری.
\subsubsection{سوختگی‌های ناشی از مایع}
 مایع به دلیل تبخیر سریع باعث سوختگی شدید سرد می‌شود. همچنین، تبخیر می‌تواند تجهیزات را به حدی سرد کند که خطر سوختگی را افزایش دهد.
 استفاده از دستکش و لباس‌های محافظ در هنگام کار با 
 \lr{LPG}
  مایع.
 عایق‌کاری مناسب تجهیزات برای جلوگیری از تماس مستقیم.
\subsubsection{احتراق مخلوط بخار/هوا در اثر نشت}
مخلوط بخار 
\lr{LPG}
و هوا می‌تواند در فاصله‌ای دورتر از نقطه نشت آتش بگیرد و شعله به منبع نشت بازگردد.بررسی منظم و رفع نشتی تجهیزات ذخیره و انتقال \lr{LPG}.
نصب شیرهای خودکار قطع گاز برای جلوگیری از گسترش شعله.
\subsubsection{خطرات مخازن خالی}
مخزن خالی
 \lr{LPG}
  ممکن است همچنان حاوی بخار
\lr{\textbf{LPG}}
تخلیه و تهویه کامل مخازن قبل از انجام هرگونه تعمیرات.
برچسب‌گذاری مخازن برای هشدار به افراد از خطرات احتمالی.
\subsection{سیستم‌های ذخیره‌سازی و مدیریت 
	\lr{LPG} 
	در کشتی‌ها}
سوخت‌رسانی
 \lr{LPG}
  به کشتی‌ها به‌عنوان یک سوخت دریایی مزایا و خطراتی دارد. 
   \lr{LPG}
   در حالت مایع خود قابل اشتعال یا انفجار نیست، نشت آن می‌تواند باعث ایجاد بخارهایی شود که به راحتی با باد پراکنده و در صورت برخورد با منبع حرارتی ممکن است آتش بگیرند. همچنین، نشت
    \lr{LPG}
    روی آب می‌تواند منجر به استخر آتش\LTRfootnote{\lr{pool fire}}
     شود که بسیار داغ‌تر و سریع‌تر از آتش‌های ناشی از نفت یا بنزین می‌سوزد و قابل خاموش شدن نیست. در عملیات سوخت‌رسانی، نشت 
    \lr{LPG}
    می‌تواند خطرات زیادی از جمله آتش‌سوزی یا انفجار در مناطق بندری ایجاد کند. به همین دلیل، تدابیر ایمنی ویژه‌ای مانند استفاده از لوله‌های دو جداره و آشکارسازهای هیدروکربنی برای جلوگیری از نشت و آسیب به کشتی‌ها ضروری است. با این حال، سوخت‌رسانی
     \lr{LPG}
     هنوز چارچوب نظارتی رسمی و دستورالعمل‌های مشخصی ندارد و توسعه این دستورالعمل‌ها می‌تواند به بهبود ایمنی، کارایی و آگاهی محیط‌زیستی کمک کند، مشابه آنچه برای دیگر سوخت‌ها مانند 
     \lr{LNG }
     انجام شده است.
\subsection{ایمنی و الزامات فنی مرتبط با حمل }
برای بهبود ایمنی و کارایی در سوخت‌گیری \lr{LPG}، لازم است استانداردها و مشخصات فنی تجهیزات مانند شیلنگ‌ها، نازل‌ها و شیرآلات تدوین شود و معیارهای عملکرد و الزامات ایمنی آن‌ها تعریف گردد. روال‌هایی برای تست و صدور گواهی تجهیزات سوخت‌گیری ایجاد شده و همراه با مستندات و مواد آموزشی جامع، در اختیار ذینفعان از جمله اپراتورها و خدمه قرار گیرد. پروژه‌های آزمایشی و نمایش‌های عملی به منظور اعتبارسنجی چارچوب‌ها و شناسایی شکاف‌ها یا مشکلات احتمالی اجرا شده و از نتایج آن برای اصلاح رویه‌ها و ارتقای ایمنی استفاده شود.

این فرآیندها تضمین می‌کند که تمامی تجهیزات و عملیات مرتبط با سوخت‌گیری \lr{LPG} مطابق با استانداردهای بین‌المللی بوده و صدور گواهی‌ها به ایمنی جهانی کمک کند.
\\
فرآیند سوخت‌گیری در سه مرحله انجام می‌شود.
\subsubsection{مرحله قبل از سوخت‌گیری }
مرحله پیش از سوخت‌گیری 
\LTRfootnote{\lr{Before bunkering}}
	 از سفارش سوخت آغاز شده و با شروع فرآیند سوخت‌گیری خاتمه می‌یابد.  
در این مرحله آمادگی، انجام تمام اقدامات لازم برای اطمینان از انتقال ایمن سوخت بسیار مهم است. این اقدامات شامل موارد زیر می‌شوند:

\begin{itemize}
	\item اطمینان از اینکه تمامی یافته‌های ارزیابی ریسک به درستی مورد توجه قرار گرفته‌اند.
	\item ارزیابی سازگاری بین کشتی دریافت‌کننده سوخت و تأسیسات .
	\item تهیه و توافق بر روی برنامه واکنش اضطراری.
	\item ارائه دستورالعمل‌های ایمنی و آموزش کارکنان .
	\item هماهنگی با نهادهای مسئول برای دریافت مجوزهای لازم.
	\item ارزیابی فرآیندهای مرتبط دیگر، مانند عملیات هم‌زمان \LTRfootnote{\lr{SIMOPS}}.
	\item تعیین جزئیات عملیاتی مانند نرخ انتقال، محدودیت‌های بارگیری،  
	خاموشی اضطراری
	\LTRfootnote{\lr{ESD}}
	، سیستم ایمنی اضطراری
	\LTRfootnote{ERS} 
	و غیره.
	\item تکمیل تمامی چک‌لیست‌های مورد نیاز پیش از سوخت‌گیری.
\end{itemize}

\subsection{مرحله سوخت‌گیری}

فرآیند سوخت‌رسانی
\LTRfootnote{During bunkering}
 با اتصال کشتی دریافت‌کننده به تأسیسات سوخت‌رسانی آغاز می‌شود و با انتقال واقعی سوخت ادامه می‌یابد، و با اقدامات لازم برای بستن ایمن شیر از تأسیسات سوخت‌رسانی خاتمه می‌یابد.  
در طول مرحله سوخت‌گیری، بخش‌های حیاتی سیستم باید به طور مداوم کنترل شوند، از جمله:

\begin{itemize}
	\item سطح مخازن؛
	\item فشار مخازن؛
	\item دمای مخازن؛
	\item نرخ انتقال پمپ؛
	\item نرخ جریان پمپ؛
	\item عملیات سیستم‌های \lr{ESD} و \lr{ERS}؛
	\item تنظیم خطوط پهلوگیری و شیلنگ‌ها؛
	\item نظارت و حفظ سایر جنبه‌های ایمنی، مانند مناطق ایمنی.
\end{itemize}

\subsection{پس‌از سوخت‌گیری}
پس از اتمام سوخت‌گیری
\LTRfootnote{After bunkering}
، باید به نکات زیر توجه شود:
\begin{itemize}
	\item انجام موفقیت‌آمیز فرآیندهای سیستمی ، مانند تبخیر خطوط و خنثی‌سازی گازها در خطوط و شیلنگ‌ها، بدون انتشار گاز به جو.
	\item قطع ایمن ارتباط بین کشتی دریافت‌کننده و تأسیسات سوخت‌رسانی.
	\item جداسازی ایمن کشتی دریافت‌کننده یا کشتی سوخت‌رسان از کشتی دریافت‌کننده و اطلاع‌رسانی به مقامات بندری.
\end{itemize}

\section{الزامات قانونی}
برای کاهش ریسک‌های مرتبط با کشتی، خدمه و محیط زیست، کدهای ایمنی بین‌المللی پیش‌نیازهای لازم برای تجهیزات، ماشین‌آلات و سیستم‌های کشتی را تعیین می‌کنند. این الزامات شامل استانداردهای عملکردی، ارزیابی ریسک، مقررات و نیازمندی‌های عملیاتی است که همراه با آموزش مناسب خدمه، ایمنی عملیات کشتی را تضمین می‌کنند.
\subsection{کد \lr{IGC}}
این کد مرتبط با ساخت و تجهیز کشتی‌هایی که گاز مایع شده را به صورت فله حمل می‌کنند و استفاده از این گازها به‌عنوان سوخت هست.
\subsection{کد \lr{IGF}}
این کد مخصوص کشتی‌های غیرحامل گاز که از گاز یا سوخت‌های با نقطه اشتعال پایین، مانند
\lr{LPG}،
به‌عنوان سوخت استفاده می‌کنند.	

\subsection{استاندارد \lr{MSC.1/Circ.1666}}
استاندارد ایمنی برای کشتی‌هایی که از ال پی جی به‌عنوان سوخت استفاده می‌کنند.
...
\subsection{استاندارد \lr{MSC.1/Circ.1679}}
...راهنمایی‌های موقت برای استفاده از محموله ال پی جی به‌عنوان سوخت.


\section{اقتصاد و بازار \lr{LPG}}
\subsection{چرخه عمر}
دستورالعمل‌های چرخه عمر گازهای گلخانه‌ای 
\LTRfootnote{\lr{LCA}}
 سازمان بین‌المللی دریانوردی  برای سوخت‌های دریایی، شامل 
 \lr{LPG}
 ، ابتدا در نشست 
 \lr{MEPC 80 (MEPC.376(80))}
 تصویب شدند و تمامی مراحل زنجیره تأمین این سوخت را پوشش می‌دهند. در نشست
  \lr{MEPC 81}،
   نسخه
    \lr{2024}
    این دستورالعمل‌ها 
  \lr{  (MEPC.391(81)) }
    با اصلاح عوامل توزیع پیش‌فرض، به‌روزرسانی الگوی عوامل توزیع از چاه به مخزن
    \LTRfootnote{Well-to-Tank}
    \cref{Suplly-Chain}
     و اضافه‌شدن الگوی جدید برای عوامل توزیع از مخزن به مصرف 
\LTRfootnote{Tank-to-Wake}
      بازنگری شد.
 \cite{Comparative-Life-Cycle}
 \begin{figure}[!h]
 	\centering
 	\includegraphics[width=15cm]{Figures/LPG/supply-chain.png}
 	\caption{ زنجیره تأمین سوخت}\label{Suplly-Chain}
 \end{figure}
 
\subsection{تجارت جهانی}
 تقاضای جهانی  
 \lr{LPG }
 در
  \cref{future}
 و تجارت دریایی مرتبط با آن در حال افزایش است. در سال 2022،
  تقاضای جهانی
   \lr{LPG}
    با رشد 
    $\%$5.3
    به رکورد 342 میلیون تن رسید. پیش‌بینی‌ها نشان می‌دهد که حجم تجارت دریایی
    \lr{LPG}
     در سال 2023 حدود 6 میلیون تن افزایش یافته و نسبت به 2022 رشد 5$\%$
   
      داشته است. همچنین، برای سال 2024، رشد
      $\%$3.2
        پیش‌بینی شده و انتظار می‌رود حجم تجارت تا سال 2027 به حدود 142 میلیون تن برسد.
 \begin{figure}[!h]
	\centering
	\includegraphics[width=15cm]{Figures/LPG/future.png}
	\caption{ درخواست \lr{LPG}}\label{future}
\end{figure} 
 
\subsection{قیمت سوخت}
استفاده از
 \lr{LPG}
 (گاز مایع) به عنوان سوخت، به‌دلیل قیمت جذاب آن در برخی مناطق مانند ایالات متحده و خلیج فارس، گزینه‌ای اقتصادی برای مالکان و اپراتورها محسوب می‌شود. از نظر هزینه‌های سرمایه‌ای 
 \LTRfootnote{(CAPEX)}،
  \lr{LPG }
 مزایای قابل توجهی نسبت به سایر گزینه‌های سوخت دوگانه 
\LTRfootnote{Dual-Fuel}
  مانند 
  \lr{LNG}
   دارد؛ به‌طوری که ساخت یک کشتی کانتینری ۱۰ هزار 
   \lr{TEU}
   با سوخت 
   \lr{LPG}
    حدود ۱۰۰ میلیون دلار هزینه دارد، یعنی ۲۰$\%$ ارزان‌تر از کشتی مشابه با سوخت 
    \lr{LNG }
    که حدود ۱۲۵ میلیون دلار هزینه می‌برد. همچنین، هزینه تبدیل موتورهای دیزلی به موتورهای دوگانه 
    \lr{LPG }
    (بین ۹.۵ تا ۲۷ میلیون دلار) در مقایسه با تبدیل به
     \lr{LNG}
     (بین ۱۲ تا ۳۳ میلیون دلار) مقرون‌به‌صرفه‌تر است. این عوامل باعث می‌شود.
      \lr{LPG }
      به‌عنوان گزینه‌ای جذاب برای صنعت کشتیرانی مطرح شود.
(\cref{price})
\begin{figure}[!h]
   	\centering
   		\includegraphics[width=15cm]{Figures/LPG/price.png}
   	\caption{ قیمت سوخت در خلیج فارس}\label{price}
\end{figure}
\\
\\
\subsection{تأمین و نگهداری}
\section{نتیجه‌گیری}
\begin{itemize}
	\item \textbf{مزایای زیست‌محیطی \lr{LPG}:}
	\begin{itemize}
		\item LPG به‌عنوان یک سوخت فسیلی مزیت‌های قابل توجهی در کاهش \textbf{آلودگی هوا} نسبت به سوخت‌های نفتی سنتی (مانند مازوت) دارد.
		\item استفاده از LPG موجب \textbf{کاهش انتشار گازهای گلخانه‌ای} می‌شود، به‌ویژه اگر با فناوری‌های مکملی مانند \textbf{کربن‌گیری در کشتی} همراه شود.
		\item این سوخت می‌تواند با \textbf{مقررات سازمان بین‌المللی دریانوردی (IMO)} درباره کاهش اکسیدهای گوگرد مطابقت داشته باشد و همچنین در بلندمدت با اهداف کربن‌زدایی این سازمان همخوانی داشته باشد.
	\end{itemize}
	
	\item \textbf{پتانسیل LPG برای آینده:}
	\begin{itemize}
		\item استفاده از LPG در بلندمدت به تولید سوخت‌های تجدیدپذیر وابسته است، که انتظار می‌رود با سرعت بالایی افزایش یابد.
		\item با رشد تجارت دریایی و افزایش تقاضا برای LPG، ناوگان جهانی کشتی‌های حمل LPG رشد خواهد کرد و این فرصت برای استفاده از LPG به‌عنوان سوخت افزایش می‌یابد.
		\item زیرساخت‌های حمل‌ونقل، ذخیره‌سازی و استفاده از LPG طی چند دهه به‌خوبی توسعه یافته است.
	\end{itemize}
	
	\item \textbf{چالش‌های موجود:}
	\begin{itemize}
		\item تکنولوژی‌های موتوری برای LPG محدود است. به‌عنوان مثال، هنوز موتور دریایی چهارزمانه‌ای که بتواند از LPG استفاده کند، وجود ندارد، بنابراین موتورهای کمکی کشتی‌ها نیاز به سوخت‌های دیگری برای کربن‌زدایی دارند.
		\item قوانین و چارچوب‌های مقرراتی برای استفاده از LPG به‌عنوان سوخت، خصوصاً در حوزه \textbf{سوخت‌رسانی (bunkering)}، هنوز کامل نیست و فقط راهنماهای اولیه در سطح IMO تدوین شده است.
	\end{itemize}

	\item \textbf{عامل تعیین‌کننده:}
	\begin{itemize}
		\item آینده LPG به‌عنوان یک سوخت مهم در صنعت دریانوردی به سرعت \textbf{کربن‌زدایی تولید LPG} و پیشرفت فناوری‌های مرتبط مانند \textbf{کربن‌گیری} وابسته است.
		\item همچنین LPG ممکن است به‌عنوان یک سوخت \textbf{انتقالی} تا زمان توسعه کامل سوخت‌های بدون کربن یا نزدیک به صفر کربن عمل کند.
	\end{itemize}
\end{itemize}



%\include{chapter3}
%\include{chapter4}
\chapter{پیشینه تحقیق}
\begin{landscape}
\begin{center}
	\begin{tikzpicture}[node distance=3cm]
		\node[state] (A) {$Q_1$};
		\node[state] (B) [below left of = A] {$Q_2$};	
	\end{tikzpicture}
\end{center}
\end{landscape}
%\usetikzlibrary{shapes.geometric, arrows}
%
%\tikzstyle{startstop} = [rectangle, rounded corners, minimum width=3cm, minimum height=1cm,text centered, draw=black, fill=red!30]
%\tikzstyle{process} = [rectangle, minimum width=3cm, minimum height=1cm, text centered, draw=black, fill=orange!30]
%\tikzstyle{arrow} = [thick,->,>=stealth]
%
%
%	
%	\begin{tikzpicture}[node distance=2cm]
%		
%		\node (Covert) [startstop] {Covert};
%		\node (Channel) [process, below of=Covert] {Channel};
%		\node (Linguistic) [process, below of=Channel] {Linguistic};
%		\node (Steganography) [process, below of=Linguistic] {Steganography};
%		\node (Technical) [process, right of=Steganography, xshift=4cm] {Technical};
%		\node (Information) [process, below of=Steganography] {Information};
%		\node (Hiding) [process, below of=Information] {Hiding};
%		\node (Anonymity) [process, below of=Hiding] {Anonymity};
%		\node (Copyright) [process, below of=Anonymity] {Copyright};
%		\node (Marking) [process, below of=Copyright] {Marking};
%		
%		\draw [arrow] (Covert) -- (Channel);
%		\draw [arrow] (Channel) -- (Linguistic);
%		\draw [arrow] (Linguistic) -- (Steganography);
%		\draw [arrow] (Steganography) -- (Information);
%		\draw [arrow] (Information) -- (Hiding);
%		\draw [arrow] (Hiding) -- (Anonymity);
%		\draw [arrow] (Anonymity) -- (Copyright);
%		\draw [arrow] (Copyright) -- (Marking);
%		\draw [arrow] (Steganography) -- (Technical);
%		
%	\end{tikzpicture}


%--------------------------------------------------------------------------appendix( مراجع و پیوست ها)
\chapterfont{\vspace*{-2em}\centering\LARGE}%

\appendix
\bibliographystyle{plain-fa}
\bibliography{references}
\include{appendix1}
%--------------------------------------------------------------------------dictionary(واژه نامه ها)
%اگر مایل به داشتن صفحه واژه‌نامه نیستید، خط زیر را غیر فعال کنید.
\parindent=0pt
%
\chapter*{واژه‌نامه‌ی فارسی به انگلیسی}
\pagestyle{style9}

\addcontentsline{toc}{chapter}{واژه‌نامه‌ی فارسی به انگلیسی}
%%%%%%
\begin{multicols*}{2}

{\bf آ}
\vspace*{3mm}


\farsiTOenglish{اسکالر}{Scalar}




\end{multicols*}%
%%%%%%
\chapter*{ واژه‌نامه‌ی انگلیسی به فارسی}
\pagestyle{style9}
\lhead{\thepage}\rhead{واژه‌نامه‌ی انگلیسی به فارسی}
\addcontentsline{toc}{chapter}{واژه‌نامه‌ی انگلیسی به فارسی}

\LTRmulticolcolumns
\begin{multicols}{2}
{\hfill\bf  \lr{A}}
%%\vspace*{1.5mm}

\englishTOfarsi{Automorphism}{خودریختی}

\vspace*{3mm}
{\hfill\bf   \lr{B}}
%%\vspace*{1.5mm}
\end{multicols}
%--------------------------------------------------------------------------index(نمایه)
%اگر مایل به داشتن صفحه نمایه نیستید، خط زیر را غیر فعال کنید.


\end{document}